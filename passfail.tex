
The Software Review Board will meet once a full run of all Test Cases has been performed, and subsequently after a complete run of all outstanding Test Cases. \textit{We don't actually have a software review board, not sure what the equivalent would be? is it an in-system, in-project or independent body?}


\subsection{Pass/Fail Criteria}

Pass/Fail criteria vary slighty depending on the type of test being performed.


A Test Case will be considered ``Passed'' when:
\begin{itemize_single}
\item All of the test steps of the Test Case are completed and
\item All open SPRs from this Test Case agreed in Software Review Board are considered noncritical.
\end{itemize_single}

A Test Case will be considered ``Partially Passed'' when:
\begin{itemize_single}
\item Only a subset of all of the test steps in the Test Case are completed but the overall purpose of the test has been met and
\item Any critical SPRs from this Test Case agreed in Software Review Board are still not closed.
\end{itemize_single}

A Test Case will be considered ``Failed'' when:
\begin{itemize_single}
\item Only a subset of all of the test steps in the Test Case are completed and the overall purpose of the test has not been met and
\item Any critical SPRs from this Test Case agreed in Software Review Board are still not closed.
\end{itemize_single}

\subsubsection{Key Performance Metrics}

\textit{Note: Given the incomplete 1:1 match between Key Performance Metrics listed in LDM-240 (which was a spreadsheet) and LSE-30 (aka OSS), we could theoretically have a situation where we pass our KPMs but fail an OSS metric. I think this is unlikely, but we would need to complete the OSS flowdown to be able to demonstate that to a skeptic. If we re-flowdown and come up with KPM 2.0s though, this section would stand as written. We just have to surrender the pedantic point that they wouldn't be ``Key'' at that point, they would be \textbf{the} performance metrics.}

Key Performance Metrics 







