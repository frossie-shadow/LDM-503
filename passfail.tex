
\section{Pass/Fail Criteria}


A Test Case will be considered ``Passed'' when:
\begin{itemize_single}
\item All of the test steps of the Test Case are completed and
\item All open SPRs from this Test Case are considered noncritical by DMCCB.
\end{itemize_single}

A Test Case will be considered ``Partially Passed'' when:
\begin{itemize_single}
\item Only a subset of all of the test steps in the Test Case are completed but the overall purpose of the test has been met and
\item Any critical SPRs from this Test Case agreed in Software Review Board are still not closed.
\end{itemize_single}

A Test Case will be considered ``Failed'' when:
\begin{itemize_single}
\item Only a subset of all of the test steps in the Test Case are completed and the overall purpose of the test has not been met and
\item Any critical SPRs from this Test Case agreed in Software Review Board are still not closed.
\end{itemize_single}

Note that in \citeds{LPM-17} science requirements are described as having a minimum specification, a design specification and a stretch goal. We preserve these distinctions where they have been made in, for example, the verification framework and automated metric harness. However for the purposes of Pass/Fail criteria, it is the design specification that is verified as having been met for a test to pass without intervention of the Software Review Board.

In the event that a requirement is failing its design specification and the minimum specification is invoked, this is an LSST project level issue and is escalated beyond the scope of this plan.
