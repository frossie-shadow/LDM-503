\begin{longtable} {|l|l|p{0.7\textwidth}|}\hline 
{\bf Date/Freq} &{\bf Location}& {\bf Title, Description} \\ \hline

Nightly &  Amazon & {\bf Nightly Tests} \newline 
Run all automated tests on all DM packages automatically. 
\\ \hline
Weekly & Amazon & {\bf Integration tests} \newline 
Basic Sanity check to make sure code compiles at no regressions have occurred and also pushing though a basic data set.
\\ \hline

TBP& NCSA & Interface tests \newline
The interface tests have to be planned and documented in a separate test plan that should include 
tests for each two parties on an interface (2by2 tests) as well as tests for all parties. Some of these will be covered again in E2E tests but before that we should be confident they work. {\bf This includes internal and external interfaces.}
\\ \hline

TBP & NCSA + IN2P3 & End to End Tests ?? Freeze software for Ops .. \url{https://confluence.lsstcorp.org/display/DM/Data+Processing+End+to+End+Testing}  What is the status of these ?
\\ \hline


F17 & NCSA & {\bf Science Platform with WISE data in PDAC}  \newline
SUIT continues PDAC development, adding the WISE data, further exercising the DAX dbserv and imgserv APIs, and taking advantage of metaserv once it becomes available
\\ \hline

F17 & NCSA& {\bf HSC reprocessing } \newline
Validate the data products withe LSST stack match or improve the HSC products - thus validating the stack.  
Validate the ops platform in NCSA. Validate some procedures like installing the stack, patches, starting, stopping production. Generate validation data set for weekly integration and other tests. 
\\ \hline

F17 &   & {\bf AP alert generation validation}\newline  
Validate AP alert generation stack performance on several DECam and HSC datasets.  Begin continuous integration testing. \\ \hline

Feb 2018 & NCSA? &  {\bf Camera DAQ Integration Test }\newline
The data acquisition hardware should be available to DM 13th Feb. We should test it.
\\ \hline

S18 & NCSA? &  {\bf AP system validation}\newline
Validate AP alert distribution and mini-broker system fed by live or simulated live data. 
\\ \hline

June 2018 & NCSA &  {\bf DM ComCam system test }\newline
ComCam will be in Tucson July 24th, the DM system must be ready to deal with it. 
\\ \hline

Aug 2018 & NCSA &  {\bf DM Camera data test }\newline
Partial camera data should be available to DM July 31st. We plan to test DM stack with it. 
\\ \hline

2018 & NCSA & {\bf Spectrograph Data acquisition } \newline
\ldots  {\color{red} Do we need a test BEFORE THIS?}
\\ \hline

Oct 2018 &  NCSA & {\bf Operations rehearsal for commissioning }
With TBD weeks commissioning (lets say a week) - pick which parts of plan we could rehearse.
Chuck suggests Instrument Signal Removal should be the focus of this (or the next rehearsal).
\\ \hline

Feb 2019 & NCSA &  {\bf DAC validation  }\newline
There is a project Milestone that DAC/DM/Networks are  available March 15th. We need to run tests in Feb to show this is ready.
\\ \hline

Oct 2019 & NCSA &  {\bf  Operations rehearsal \#2 for commissioning} 
More complete rehearsal - where do the scientist look at quality data? How do they feed it back to the Telescope ?
How do we create/update calibrations ? Exercises some of the control loops.
\\ \hline

Jan 2020 & Base  &  {\bf  Operations rehearsal \#3 for commissioning} 
Dress rehearsal - Just like it will be April for the actual commissioning.
\\ \hline

Dec 2020 &  NCSA &  {\bf Operations  Rehearsal Data Release (Commissioning Data)}
\\ \hline

March 2021 &  Base &  {\bf DM Software for Science Verification Test}
Science verification starts April, DM software must be installed and validated prior to start of Science Verification (which should be called validation perhaps)
\\ \hline

2021 &  NCSA &  {\bf Operations  Rehearsal Data Release (Regular Data).}
\\ \hline

Feb 2022 &  NCSA/Base &  {\bf Operations  Rehearsal(s)}
Rehearsals for real operations which start Oct 2022
\\ \hline
Sept 2022 &  NCSA/Base &  {\bf Operations  Rehearsal(s)}
Full Dress rehearsal for real operations which start Oct 2022
\\ \hline

\hline

\end{longtable}
