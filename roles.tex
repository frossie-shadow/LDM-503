

\section{Roles and Reporting}

Each test specification must make clear who the \emph{tester} is.

Testers report issues (SPRs) through the \product\ ticketing system (e.g.\ JIRA at the time of this document revision) and also write a test report (and/or provide any necessary configuration for automatic report generation).

The test reports will be used to populate the verification control document (see \secref{sect:approach}). We are monitoring the LSST Systems Engineer's approach to plan commissioning tests for LSST system-wide verification and will evaluate the merits of using the same toolchain for \product\ verification.

Operations rehearsals require an \emph{ops rehearsal coordinator} to oversee the process rather than tests. The rehearsal may not be directed by the Operations Manager since that person has a major role in the rehearsal. An individual not involved in the rehearsal itself will be identified to perform this function.

Tests and procedures will sometimes fail -- a test specification may be rerun several times until it passes but the report must include an explanation than indicates that any failures were understood (e.g.\ they were due to a fault that was fixed) or repeated sufficient times to ensure that passing the test was not transient success.

For large scale tests and rehearsals the DMCCB, or an individual designated by it, will be tasked to write up the findings as well as decide on timescales for re-running part or all of a test in case of failure or partial success.

Other parties that have a relevant role in \product\ verification are identified in the appropriate sections of the document; these are involved in their primary capacity (e.g.\ the DM Systems Engineer) and so are not individually listed in this section.
