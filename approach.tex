\section{DM Verification Approach \label{approach}}

Our approch towards verifying the \product\ requirements follows
standard engineering practice.  Each high level component will have at
least one test specification defining a set of tests related to the
design requirements for the component.  These specifications are
represented on the top of \figref{fig:doctree}. Any given requirement
may have several tests associated with it in the specification; the
tests may be phased to account for incremental delivery depending on
the need for certain functionality at a specific time.

The test spec will cover all aspects of the test as outlined in
\secref{sect:tsform}. These high level test specifications may call
out individual lower level test specification where it makes sense
(either technically or programatically) to test lower-level components
in isolation.

\subsection{Reports}

As we execute tests we will generate test reports on the Pass/Fail of the individual tests related to specific requirements. 
This Information will allow us to build a Verification Control Document (VCD) (right of \figref{fig:doctree}).
The VCD will provide a \% verification of each requirement in DM (rolled up to OSS requirements also).
\figref {fig:doctree} currently calls for a report from each test spec - this may be captured directly in e.g.\ Jira and does not need to originate as a document-format (Word, LaTex) source.

In cases of reports that are generated via automantic (continuous) verification, the report may be in the format of a Jupyter Notebook that simultaneously can serve as test specification and test report and, in some cases, the test script itself. This is the preferred method provided the notebook-as-report is satisfactorily captured in Docushare. 


\begin{figure}
\begin{center}
 \includegraphics[angle=-90,width=0.7\textwidth]{images/DocTree}
 \caption{Documentation tree for DM software relating the high level documents to each other. (from \citeds{LDM-294}\label{fig:doctree}}

 \end{center}
 \end{figure}

 The DM components are  outlined in \citeds{LDM-294} and detailed in \citeds{LDM-148}. At a high level these components are represented in figure \figref{fig:dmsdeploy}.  Based on those components we can see the set of Test Specifications needed in \tabref{tab:testspecs}. That table does not contain all of the document numbers yet for cases for second-level components (and in some cases they may be rolled up into their parent). 


\begin{figure}[htbp]
	\begin{center}
		\includegraphics[width=0.8\textwidth]{images/DMSDeployment}
		\caption{DM components as deployed during Operations. Where components are
			deployed in multiple locations, the connections between them are labeled with
			the relevant communication protocols. Science payloads are shown in blue.
		\label{fig:dmsdeploy} (from \citeds{LDM-148})}
	\end{center}
\end{figure}

\subsection{Test Items}

\begin{table}
	\caption{Components from LDM-148 with the test specifications to verify them. \label{tab:testspecs}}
	%%%%%%%%%%%%%%%%%%%%%%%%%%%%%%%%%%%%%%%%%%%%%%%%%%%%%%%%%%%%%%%%%%%%%%%%%%%%%%%%%%%%%%%%%%%%%%%%%
%%  Test Spec  table generated by makeTestSpecTable.py do not modify.
%%%%%%%%%%%%%%%%%%%%%%%%%%%%%%%%%%%%%%%%%%%%%%%%%%%%%%%%%%%%%%%%%%%%%%%%%%%%%%%%%%%%%%%%%%%%%%%%%
\begin{longtable}{|p{0.3\textwidth}|p{0.2\textwidth}}\hline 
 \bf Component & Testing Spec \\ \hline   
NCSA Enclave &  LDM.. \\ \hline 
L1 System &  LDM-??? \\ \hline 
L1 Prompt Processing &  ??? \\ \hline 
L1 Alert Distribution &  ??? \\ \hline 
L1 Alert Filtering (mini Broker) &  ??? \\ \hline 
L1 Quality Control &  ??? \\ \hline 
L1 OCS Batch Processing &  ??? \\ \hline 
L1 Offline Processing &  ??? \\ \hline 
L2 System &  LDM-??? \\ \hline 
L2 QC &  ??? \\ \hline 
L2 Data Release Production &  ??? \\ \hline 
L2 Calibration Products Production  &  ??? \\ \hline 
Data Backbone &  LDM.. \\ \hline 
DBB Data Services &  LDM-??? \\ \hline 
DBB QSERV &  ??? \\ \hline 
DBB Databases &  ??? \\ \hline 
DBB Image Database/Metadata Prov &  ??? \\ \hline 
DBB Data Butler Client &  ??? \\ \hline 
DBB infrastructure &  LDM-??? \\ \hline 
DBB Tape Archive &  ??? \\ \hline 
DBB Cache &  ??? \\ \hline 
DBB Data Endpoint &  ??? \\ \hline 
DBB Data Transport &  ??? \\ \hline 
Networks  &  ??? \\ \hline 
Base Enclave &  LDM.. \\ \hline 
Prompt Processing Ingest  &  ??? \\ \hline 
Telemetry Gateway &  ??? \\ \hline 
Image and EFD Archiving &  ??? \\ \hline 
OCS Driven Batch Control &  ??? \\ \hline 
Data Access Center Enclave &  LDM.. \\ \hline 
 Bulk Data Distibution &  ??? \\ \hline 
Science Platform &  LDM-??? \\ \hline 
Science Platform JupyterLab &  ??? \\ \hline 
Science Platform Portal &  ??? \\ \hline 
DAX VO+ Services &  ??? \\ \hline 
Commisioning Cluster Enclave &  LDM- \\ \hline 
\end{longtable} 

\end{table}

The test items covered in this test plan are:

\begin{itemize_single}
\item \product \ and its primary components for testing and integration purposes. These are listed in Table \ref{tab:testspecs}. All components listed in orange and yellow have specifications in the corresponding documents listed. Major sub-components in white may have individual test specifications or be addressed in the component they are under depending on applicable factors such as whether they are scheduled for testing at the same time and/or whether they share architectural components or are largely distinct. 

\item The external interfaces between \product and other sub-systems. These are described in [Docushare collection]

\item Operational procedures like Data Release Process, the Software Release Process and the Security Plan. 

\end{itemize_single}


  
\subsection{Testing Specification document format}\label{sect:tsform}

The Testing Specification documents are drawn in conjunction with the LSST System Engineer. In all cases they include:

\begin{itemize_single}

\item A list of components being tested within the scope of the Test Specification Document. 

\item A list of features in those components that are being explicitly tested.

\item How those features related to identified requirements for that component
  
\item A description of the environment in which the tests are carried out (eg. hardware platform) and a description of how they differ from the operational system in tests prior to final integration (eg. interfaces that may be mocked without affecting that component's testing). 
  
\item The inputs (eg. data, API load) that are to be used in the test

\item Pass-fail criteria (eg. metrics to be met) 

\item How any outputs that are used to determine pass/fail (eg. data or metrics) are published (made available). 

\item A Software Quality Assurance manifest listing (as relevant) code repositories, configuration information, release/distribution methods and applicable documentation (such as installation instructions, developer guide, user guide etc.)
  
\end{itemize_single}


