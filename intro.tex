
\section{Introduction \label{sect:intro}}
In this document  we outline the verification and validation approach for LSST Data Management. In addition we outline some of the high level test milestones. 


\subsection{Objectives \label{sect:objectives}}

We describes the test and verification approach for DM and describe  constraints and limitations in the testing to be performed. 
We also describe the validation tests to be performed on the partially and fully integrated system. 
We do not describe all tests in details but leave that to dedicated test plans.

\subsection{Scope \label{sect:scope}}

This provides the approach and plan for all od DM. It covers interfaces to DM but nothing outside of DM. 
This document will be updated in response to any  requirements updates.

\subsection{Assumptions}  
 We will run large scale Science Validations. A large amount of informal science validation will be done in the the teams and documented in technical notes, in this test plan we are looking for broad validation and specifically {\em operaability} i.e. can we run this system everyday for a long period of time (years).

\subsection{Applicable Documents \label{sect:ad}}
When applicable documents change a change may be required in this document.
\begin{tabbing}
AUTH-NUM\= \kill 
\citeds{LPM-55}\>	LSST Quality  Assurance Plan \\
\citeds{LDM-294} \>	DM Project Management Plan   \\
\citeds{LDM-148}\>	DM Architecture\\
% perhaps \citell{LL:AUTH-code}\>	Software Requirements Specification for \CU,\\
\end{tabbing}

\subsection{References}

\renewcommand{\refname}{}
\bibliography{lsst,gaia_livelink_valid,refs,books,refs_ads}

\subsection{Definitions, acronyms, and abbreviations \label{sect:acronyms}} 
% include acronyms.tex generated by the acronyms.csh (GaiaTools)
The following table has been generated from the on-line Gaia acronym list:
\newline\newline%decrement table counter so table sin doc start at 1.
\addtocounter{table}{-1}
\begin{longtable}{|l|p{0.8\textwidth}|}\hline 
\textbf{Acronym} & \textbf{Description}  \\\hline
CU&Coordination Unit (in DPAC) \\\hline
DPAC&Data Processing and Analysis Consortium \\\hline
DPC&Data Processing Centre \\\hline
OF&Object Feature (source packet) \\\hline
SP&Software Product \\\hline
SPR&Software Problem Report \\\hline
SRS&Software Requirements Specification \\\hline
STP&Software Test Plan \\\hline
STS&Star Tracker System \\\hline
SVN&SubVersioN \\\hline
\end{longtable} 





